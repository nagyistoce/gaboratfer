\documentclass{article}
\usepackage[utf8]{inputenc}
\usepackage[croatian]{babel}
\usepackage[T1]{fontenc}
\usepackage{lmodern}
\usepackage{algorithmic}
\usepackage{algorithm}
\usepackage{longtable}
\usepackage{graphicx}
\usepackage{booktabs}
% Da bi se promjenio stil citiranja umjesto:
% [authoryear, round]
% staviti:
% [numbers, square]
\usepackage[authoryear, round]{natbib}
\usepackage{amsmath}
\usepackage{subfig}
\usepackage{fixltx2e}
\usepackage{todo}
\usepackage{url}

\begin{document}
\title{Izlučivanje značajki lica Gaborovim filterom}
\author{Tomislav Reicher \and Krešimir Antolić \and Igor Belša \and Marko Ivanković \and Ivan Krišto \and Maja Legac \and Tomislav Novak}

\maketitle

\tableofcontents

\section*{Za raspraviti}
\begin{itemize}
  \item Je li naslov ok? Je li ok spomenuti ``raspoznavanje lica'' ili ``lice''
  u samom naslovu?
\end{itemize}

\section{Uvod}
Sustav za raspoznavanje uzoraka se može opisati dijagramom:
\begin{center}
Uzorak $\rightarrow$ \fbox{Preprocesiranje} $\rightarrow$ \fbox{Izvlačenje
značajki} $\rightarrow$ \fbox{Redukcija dimenzije} $\rightarrow$
\fbox{Klasifikacija} $\rightarrow$ Razred\FUJ{Ovo pretvoriti u normalan
dijagram!}
\end{center}
Prva nezaobilazna komponenta je izvlačenje značajki za što je odabran Gaborov
filter.

Osnovna motivacija za korištenje Gaborovog filtera je veza sa biološkim osobinama
vida kod sisavaca čiji su receptori osjetljivi na orijentaciju te imaju
karakteristične prostorne frekvencije. Gaborov filter može iskoristit vizualne
osobine kao što su lokalizacija prostora, selekcija orijentacije i karakteristike
prostorne frekvencije
\citep{bhuiyan2007onfacerecognition}\nocite{daugman1985uncertainty}.\TODO{Prijevod
mi glupo zvuči\ldots Original je u sourceu, zakomentiran točno ispod ovog teksa.
Molim, neka netko provjeri prijevod. Btw.~što znači ``spatial''? Je li to
``prostorni''?}
% The principal motivation to use Gabor filters is biological relevance that the
% receptive field profiles of neurons in the primary visual cortex of mammals are
% oriented and have characteristic spatial frequencies. Gabor filters can exploit
% salient visual properties such as spatial localization, orientation selectivity,
% and spatial frequency characteristics.

% Ovo je opis o čemu koji odjeljak govori. Ovo se MORA nalaziti na kraju uvoda. 
U 2.~odjeljku prikazan je matematički model dvodimenzionalnog gaborovog filtera,
u 3.~odjeljku objašnjen je način izvlačenja i način interpretacije značajki te
način korištenja dobivenih značajki u sustavima za raspoznavanje uzoraka.
4.~odjeljak prikazuje rezultate primjene gaborovog filtera na neke od uzoraka, a
5.~navodi objašnjenja pojedinih parametara i njihov utjecaj na krajnji rezultat.
Zaključak je dan u 6.~odjeljku.\TODO{Prešturo\ldots Doraditi nakon što budemo
znali što ćemo uopće pisati\ldots}

\section{Dizajn gaborovog filtera}
Osnovna funkcijska forma 2D Gaborovog filtera definirana u prostornoj i
prostorno--ferkvencijskoj domeni određena je sa \citep{huang2005robust}
\begin{equation}
g(x,y)=\frac{1}{2\pi \sigma^2_{xy}}e^{-\left ( \frac{x'^2 +
y'^2}{2\sigma^2_{x,y}} \right)} \times \left ( e^{2\pi i r_0 x'} -
e^{-\frac{r_0^2}{2\sigma^2_{uv}}}\right),
\label{2d-gabor}
\end{equation}
pri čemu su
\begin{eqnarray*}
x' = x \cos \theta + y \sin \theta, \\
y' = -x \sin \theta + y \cos \theta,
\end{eqnarray*}
gdje je $\sigma_{xy}$ standardna devijacija Gaussove omotnice\TODO{Je li ok
ovako prevesti \emph{Gaussian envelope}?}koja karakterizira prostorni obujam i
širinu\TODO{U originalu \emph{bandwidth}. To je ok?}filtera. Parametri ($u_0$,
$v_0$) definiraju prostornu frekvenciju sinusoidalnog vala u ravnini koji
također može biti izražen polarnim koordinatama kao raidalna\TODO{U
originalu \emph{radial}}frekvencija $r_0$ i orijentacija $\theta$:
\begin{eqnarray}
r_0^2 = u_0^2 + v_0^2, \\
\tan \theta = \frac{v_0}{u_0}.
\end{eqnarray}
Frekvencija i odabir orijentacije Gaborovog filtera su izražajnije u domeni
frekvencijskog prikaza predstavljenog jednadžbom (\ref{gabor-frek}) koja
određuje koliko filter utječe na svaku frekvencijsku komponentu ulazne slike.
\begin{equation}
G(u,v) = \exp \left ( - \frac{(u-u_0)^2 + (v-v_0)^2}{2\sigma^2_{uv}}\right ) -
\exp \left ( - \frac{r_0^2}{2\sigma^2_{uv}} \right),
\label{gabor-frek}
\end{equation}
\begin{equation}
\sigma_{uv} = \frac{1}{2\pi \sigma_{xy}}.
\end{equation}
Osobina Gaborovog filtera definirana je radijalnom frekvencijom $r_0$,
orijentacijom i širinom filtera.

\section{Izvlačenje značajki pomoću gaborovog filtera}
Gaborove značajke se dobivaju konvolucijom klizečeg prozora slike i Gaborovog
filtera\ldots

\subsection{Preprocesiranje uzoraka}

\section{Primjer primjene}

\section{Utjecaj pojedinih parametara}

\section{Zaključak}

\bibliography{literatura}
\bibliographystyle{plainnat}

\newpage
\appendix
\section{\LaTeX~playground}
Ovdje se možete igrati sa \LaTeX--om. Ideja je da igranjem u ovome dijelu
naučite nešto korisno oko \LaTeX--a tako da vam oni koji znaju \LaTeX~tu ostave
koji koristan primjer uporabe. Uglavnom, uvijek se igrajte s \LaTeX--om, jer igranje bez \LaTeX--a nije sigurno igranje. 

\subsection{O novim redovima}
Dokument dijelimo na paragrafe. Tekst unutar paragrafa se ne razlama. Ilitiga,
ako napišete:
\begin{verbatim}
Želim ovo u prvom,
ovo u drugom,
a ovo u trećem, i da je blizu udaljena od       daleko.
\end{verbatim}
Dobiti će te:
\emph{Želim ovo u prvom,
ovo u drugom,
a ovo u trećem, i da je blizu udaljena od       daleko.}

Ako želite novi red unutar paragrafa, morat će te dodati
\verb|\\| na kraj red ili komandu \verb|\newline|. I to je ružno\ldots

Ako želite preći u novi paragraf dovoljno je ostaviti jedan prazan red između
prošlog i novog paragrafa, tj.:
\begin{verbatim}
Prvi paragraf.

Drugi paragraf.
\end{verbatim}

\subsection{Korištenje TODO i FUJ naredbi}
Za potrebe ovog rada, dodane su \verb|\TODO{}| i \verb|\FUJ{}| naredbe. Služe
da bi popljuvali\FUJ{Kako ružna riječ}nečije djelo ili napisali što još treba
napraviti\TODO{Navedi primjer korištenja! :D}.

U argumente ovih naredbi, tj. sadržaj \textsf{TODO}--a možete ugurati sam kod.
Npr.~recimo.\TODO{\LaTeX, ovo je \emph{naglašeno}; ili
malo matematike $\iint_a^b{x^2dx}$} No, ovdje neće proći stvari kao više
linijski programski kod i sl.

\subsection{O pisanju matematičkih izraza}
Ako želite naučiti koristiti \LaTeX~za ono što ovaj podnaslov spominje, trebati
će vam neki tutorial ili knjiga (a ima ih masu\ldots i to besplatnih!).
Ja ću samo navesti par osnova.

Primjer izraza koji se nalazi u posebnom bloku, i još je centriran:
$$x_{1,2} = \frac{-b \pm \sqrt{b^2-4ac}}{2a}$$
Napisao sam: \verb|$$x_{1,2} = \frac{-b \pm \sqrt{b^2-4ac}}{2a}$$|

Ovo je jedan način zapisa. Mogli smo koristiti:
\begin{equation}
x_{1,2} = \frac{-b \pm \sqrt{b^2-4ac}}{2a}
\label{rjesenje-kvadratne-jed}
\end{equation}
Napisao sam:
\begin{verbatim}
\begin{equation}
x_{1,2} = \frac{-b \pm \sqrt{b^2-4ac}}{2a}
\label{rjesenje-kvadratne-jed}
\end{equation}
\end{verbatim}

Na ovaj drugi način izrazi se numeriraju i možemo se lako referencirati na
njih, npr.~(\ref{rjesenje-kvadratne-jed}). Referenciramo se ne bilo koji
\verb|\label{}| pomoću naredbe \verb|\ref{}|,
npr.~\verb|\ref{rjesenje-kvadratne-jed}|.

Evo neka stranica sa par zanimljivih primjera koje najvjerojatnije nikad nećete
imati priliku primjeniti, \url{http://www.personal.ceu.hu/tex/cookbook.html} te
još malo filozofije o svemu tome
\url{http://www.math.uiuc.edu/~hildebr/tex/displays.html}.

Btw.~ako gledate source, možda se pitate zašto uvijek stavljam ``\textasciitilde{}''
nakon točke. Radi se o tome da \LaTeX~nakon svake točke stavi malo više
razmaka, jer misli da se radi o početku nove rečenice. Ako mu stavimo tildu,
onda taj razmak bude onakav kakav bi trebao biti.

\subsection{O literaturi}
Literaturu izvlačite sa citeseerxa ili google schoolara jer vam oni odmah daju
i bibtex članka koji samo kopirate u file \emph{literatura.bib}.

Primjer citiranja: \verb|\citep{Yang04facerecognition}|.

\end{document}
