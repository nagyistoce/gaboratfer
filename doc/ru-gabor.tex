\documentclass{ru}

%\usepackage{algorithmic}
%\usepackage{algorithm}
\usepackage{longtable}
\usepackage{hyperref}
\usepackage{todo}
\usepackage{hyperref}
\usepackage{textcomp}
\usepackage{float}

\hypersetup{colorlinks=true,urlcolor=blue}


\begin{document}
\title{Izlučivanje značajki lica Gaborovim filtrom}
\author{Tomislav Reicher \and Krešimir Antolić \and Igor Belša \and Marko Ivanković \and Ivan Krišto \and Maja Legac \and Tomislav Novak}
\date{20.01.2010.}
\maketitle

\newpage

\tableofcontents

\chapter{Uvod}
Raspoznavanje uzoraka je znanstvena disciplina iz područja računarskih znanosti
čiji je cilj klasifikacija ili razvrstavanje objekata u jedan od brojnih razreda
ili klasa. Iako su područja uporabe brojna u ovom radu koncentracija je na
raspoznavanju vizualnih uzoraka. Točnije, radi se o raspoznavanju lica.

Raspoznavanje lica uključuje računalno prepoznavanje identiteta na temelju
značajki dobivenih obradom slike lica. Iako ljudima lak zadatak, prepoznavanje
lica i njihovo raspoznavanje je veoma zahtjevan posao za računalo. Zadatak
postaje tim zahtjevniji ako su lica slikana pod različitim osvjetljenjem,
različitim kutovima ili osobe na slikama imaju različite izraze lica.

Kao osnovna motivacija za korištenje Gaborovog filtra za izvlačenje značajki je
veza sa biološkim osobinama vida kod sisavaca čiji su receptori osjetljivi na
karakteristične prostorne frekvencije te njihovu orijentaciju. Gaborov filtar
može iskoristit vizualne osobine kao što su lokalizacija prostora i selekcija
orijentacije pojedine prostorne frekvencije te tako omogućiti lakše
raspoznavanje objekata vanjskog svijeta
\citep{bhuiyan2007onfacerecognition}\nocite{daugman1985uncertainty}.

% Ovo je opis o čemu koji odjeljak govori. Ovo se MORA nalaziti na kraju uvoda.
U 2.~poglavlju ukratko je opisana osnovna ideja korištenja Gaborovog
filtra u izlučivanju značajki za klasifikaciju lica, u 3.~poglavlju prikazan je
matematički model dvodimenzionalnog Gaborovog filtra te su navedena objašnjenja
i utjecaj pojedinih parametara Gaborovog filtra. U 4.~poglavlju opisan je
postupak izlučivanja značajki Gaborovim filtrom te su navedeni korišteni
parametri. U 5.~poglavlju opisana je metoda kojom su lica klasificirana na temelju
izlučenih značajki, dok je provedena evaluacija opisana u poglavlju
6. Konačno, na kraju je dan zaključak.

\chapter{Teorijska razrada}
Slike lica osoba koje računalni sustav treba raspoznati podložne su različitom
šumu kao što su varijacija osvjetljenja, položaj lica, izražaji lica (lice s
osmijehom, ozbiljno lice), brkovi i brada kod lica muškaraca, naočale itd. Jedna
od metoda izlučivanja značajki koja se pritom koristi kako bi se doskočilo
navedenim problemima je izlučivanje značajki Gaborovim filtrom. Osnovna ideja u
izlučivanju značajki Gaborovim filtrom je pronaći značajke lica karakterizirane
prostornom frekvencijom \engl{Spatial frequency}, prostornom lokalizacijom
\engl{Spatial locality} i orijentacijom koje će predstavljati diskriminatorne
značajke za raspoznavanje lica različitih osoba. Promatranje slike u
frekvencijskoj domeni \engl{Frequency domain}, gdje je cijela slika predstavljena
skupom različitih prostornih frekvencija, omogućuje lakšu manipulaciju različitim informacijama sa slike. Uklanjanjem nekog manjeg skupa frekvencija i
prikazom dobivene slike moguće je ukloniti različiti šum na slici. Problem kod
prikaza u frekvencijskoj domeni je globalni karakter prikaza, tj.~za
pojedinu frekvenciju nije moguće odrediti gdje se prostorno na slici ona pojavila.
Moguće je samo utvrditi da ona na slici postoji. Gaborov filtar rješava
upravo navedeni problem prostorne lokalizacije. Pomoću njega moguće je efikasno
odrediti da li se u pojedinom dijelu slike pojavila određena frekvencija, koja
je amplituda i faza te frekvencije te koji je smjer njenog rasprostiranja.

Nakon izlučivanja skupa značajki pomoću Gaborovog filtra lice sa slike raspoznaje
se klasifikatorom. Izbor klasifikatora koji se pri tome može koristiti je
proizvoljan, a klasifikator koji je korišten u ovom radu je stroj s
potpornim vektorima \engl{Support vector machine, SVM} uz nelinearno
preslikavanje radijalnom baznom funkcijom.

\chapter{Gaborov filtar}

Jezgra Gaborovog filtra dvodimenzionalna je Gaborova funkcija koju čini
kompleksni sinusoidalni val moduliran Gaussovom funkcijom. Gaborova funkcija
dana je kao \citep{petkovgabor}:
\begin{equation}
% g(x,y)=\frac{1}{2\pi \sigma^2_{xy}}e^{-\left ( \frac{x'^2 +
% y'^2}{2\sigma^2_{x,y}} \right)} \times \left ( e^{2\pi i r_0 x'} -
% e^{-\frac{r_0^2}{2\sigma^2_{uv}}}\right),
g_{\lambda,\theta,\sigma,\gamma}(x,y) = \exp\left ( -
\frac{x'^2+\gamma^2 y'^2}{2\sigma^2}\right ) \exp \left ( j2\pi
\frac{x'}{\lambda} \right ),
\label{2d-gabor}
\end{equation}
pri čemu su
\begin{eqnarray*}
x' = x \cos \theta + y \sin \theta, \\
y' = -x \sin \theta + y \cos \theta.
\end{eqnarray*}

Ovdje $\sigma$ označava standardnu devijaciju Gaussove funkcije u $x$ smjeru,
$\gamma$ označava omjer devijacije Gaussove funkcije u $x$ i $y$ smjeru ($\gamma =
\frac{\sigma_x}{\sigma_y}$), $\lambda$ je valna duljina potpornog sinusoidalnog
vala, a $\theta$ kut smjera rasprostiranja sinusoidalnog vala.

Frekvencija i odabir orijentacije Gaborovog filtra su izražajnije u
domeni frekvencijskog prikaza predstavljenog Gaborovom funkcijom (\ref{gabor-frek}) koja
određuje koliko filtar utječe na svaku frekvencijsku komponentu ulazne slike.
\begin{equation}
G(u,v) = \exp \left ( - \frac{(u-u_0)^2 + (v-v_0)^2}{2\sigma^2_{uv}}\right ),
\label{gabor-frek}
\end{equation}
\begin{equation}
\sigma_{uv} = \frac{1}{2\pi \sigma}.
\end{equation}
Parametri ($u_0$, $v_0$) definiraju prostornu frekvenciju sinusoidalnog vala u
ravnini koja također može biti izražena polarnim koordinatama kao radijalna
frekvencija $r_0$, odnosno valna duljina $\lambda$, i orijentacija $\theta$ čime
dobivamo parametre ovisne o frekvenciji ekvivalentne onima iz Gaborove funkcije u prostornoj
domeni:
\begin{eqnarray}
r_0 = \sqrt{u_0^2 + v_0^2}, \\
\tan \theta = \frac{v_0}{u_0}, \\
\lambda = \frac{1}{r_0}.
\end{eqnarray}

Iz gornje formule možemo vidjeti da je Gaborova funkcija u frekvencijskoj domeni
jednaka Gaussovoj funkciji postavljenoj u točki $(u_0, v_0)$, odnosno $(r_0,
\theta)$, te s time na umu Gaborov filtar možemo gledati kao pojasno propusni
filtar koji propušta frekvencije u malom pojasu oko centralne frekvencije $(u_0,
v_0)$.

Kako bi bilo moguće odabrati parametre Gaborovog filtra koji će služiti
za izlučivanje značajki potrebno je promotriti utjecaj pojedinih parametara na
Gaborovu funkciju.

\section{Valna duljina ($\lambda$)}
Valna duljina se odnosi na valnu duljinu sinusoidalnog vala u Gaborovoj funkciji
kojom se ujedno definira i prostorna frekvencija na koju će sam Gaborov filtar
biti osjetljiv. Valna duljina ovdje određuje duljinu ciklusa u slikovnim
elementima \engl{Pixel} i realan je broj veći ili jednak $2$. Prema
Nyquist--Shannonovom teoremu uzorkovanja signal koji sadrži frekvencije veće od
polovine frekvencije uzorkovanja ne može biti rekonstruiran u potpunosti, stoga
je gornja granica frekvencije koju $2D$ slika može sadržavati jednaka $F_{max} =
0.5$ ciklusa/slikovnom elementu, a to upravo znači da će minimalna valna duljina
biti veća ili jednaka od $2$ slikovna elementa.

Slike realnih dijelova Gaborovih filtara uz parametre $\theta = 0,\: b
= 1$, i $\gamma = 0.5$ mogu se vidjeti na slici \ref{fig:filter-wavelengths}.

\begin{figure}[h!tb]
\centering
\subfloat[Filtar uz $\lambda = 5$.]{
\label{fig:filter-lambda-5}
\includegraphics[width=4cm]{images/wavelength5.jpg}
}
\hspace{50pt}
\subfloat[Filtar uz $\lambda = 10$.]{
\label{fig:filter-lambda-10}
\includegraphics[width=4cm]{images/wavelength10.jpg}
}
\caption{Gaborov filtar uz različite valne duljine.}
\label{fig:filter-wavelengths}
\end{figure}

% TODO: Pogledati o čemu se ovdje radi\ldots Gledao sam formulu, ništa pametno
% nisam zaključio, generirao sam slike, također ništa pametno\ldots Nešto mi
% promiče\ldots Zakomentirao sam tu rečenicu dok mi ne postane smislena.
%Uz $\lambda = 2$, ne bi se smjela koristiti kombinacija $\varphi = \pm 90$ jer
%u tom slučaju Gaborova funkcija je uzrokovana u svojim
%nul--prijelazima\TODO{``sampled in its zero crossings''?}.


\section{Orijentacija ($\theta$)}
Orijentacija određuje smjer rasprostiranja moduliranog sinusnog signala odnosno
kut normale paralelnih pruga Gaborovog filtra. Za potpuno
specificiranje prostorne frekvencije, ukoliko se ona prikazuje u polarnom
obliku odnosno kao radijalna frekvencija, nije dovoljna samo informacija o
valnoj duljini, već je potrebno odrediti i smjer u kojem se ta frekvencija
rasprostire. Orijentacija je određena kutom od $0$ do $360$ stupnjeva odnosno
$0$ do $2\pi$ radijana. Prikaze realnog dijela Gaborovog filtra sa orijentacijama od 45\textdegree, 80\textdegree\,i 0\textdegree\, možete vidjeti na slikama
\ref{fig:filter-orientation-45}, \ref{fig:filter-orientation-80} i \ref{fig:filter-lambda-10}.


\begin{figure}[h!tb]
\centering
\subfloat[Filtar uz $\theta = 45$\textdegree.]{
\label{fig:filter-orientation-45}
\includegraphics[width=4cm]{images/orientation45.jpg}
}
\hspace{50pt}
\subfloat[Filtar uz $\theta = 80$\textdegree.]{
\label{fig:filter-orientation-80}
\includegraphics[width=4cm]{images/orientation80.jpg}
}
\caption{Gaborov filtar uz različite orijentacije.}
\label{fig:filter-orientations}
\end{figure}


\section{Omjer dimenzija ($\gamma $)}
Parametar koji se preciznije naziva prostorni omjer dimenzija, određuje
eliptičnost Gaborove funkcije tj.~odnos između devijacija Gaussove funkcije u
$x$ i $y$ smjeru. Za $\gamma = 1$ eliptičnost se svodi na krug, dok je za
$\gamma < 1$ funkcija je izdužena u smjeru paralelnom s paralelnim prugama
funkcije. Primjer filtra sa različitim omjerima dimenzija se može vidjeti na
slici \ref{fig:filter-ratios}.

\begin{figure}[h!tb]
\centering
\subfloat[Filtar uz $\gamma = 0.5$.]{
\label{fig:filter-gamma-05}
\includegraphics[width=4cm]{images/wavelength10.jpg}
}
\hspace{50pt}
\subfloat[Filtar uz $\gamma = 1$.]{
\label{fig:filter-gamma-1}
\includegraphics[width=4cm]{images/ratio1.jpg}
}
\caption{Gaborov filtar uz različite omjere dimenzija.}
\label{fig:filter-ratios}
\end{figure}

\section{Standardna devijacija ($\sigma$)}
Standardna devijacija Gaussove funkcije utječe na dio prostora u kojem će se
računati odziv Gaborovog filtra, a jednako tako utječe i na širinu frekvencijskog
pojasa kojeg će Gaborov filtar propustiti, stoga je poželjno da omjer standardne
devijacije i radijalne frekvencije Gaborove funkcije bude konstantan. Gaborova
funkcija s većom frekvencijom će tako imati manju standardnu devijaciju te time
pokrivati manju površinu u prostornoj domeni od Gaborove funkcije s manjom
frekvencijom.


Širina Gaborovog filtra $b$ povezana je s omjerom
$\frac{\sigma}{\lambda}$, ($\lambda = \frac{1}{r_0}$) i definirana kao
\begin{eqnarray}
b = \log_2{\left ( \frac{\frac{\sigma}{\lambda}\pi + \sqrt{\frac{\ln2}{2}}}
{\frac{\sigma}{\lambda}\pi - \sqrt{\frac{\ln2}{2}}} \right )}, \\
\frac{\sigma}{\lambda} =
\frac{1}{\pi}\sqrt{\frac{ln2}{2}}\cdot\frac{2^b+1}{2^b-1}.
\end{eqnarray}
Prilikom izvedbe filtra vrijednost $\sigma$ se ne može direktno
odrediti, nego se mijenja samo preko vrijednosti širine filtra,
$b$. Uz zadani $b$ i poznati $\lambda$, $\sigma$ se određuje prema gornjoj
formuli.
 
Širina filtra zadana je kao pozitivni realni broj. Što je širina manja,
$\sigma$ je veća i povećava se broj naglašavajućih i prigušavajućih pruga
Gaborovog filtra.~ Primjere realnog dijela filtra s
različitim širinama moguće je vidjeti na slikama \ref{fig:filter-bandwidth-05}
i \ref{fig:filter-bandwidth-2}. Na prethodnim primjerima korištena je širina
filtra od $b = 1$.

\begin{figure}[h!tb]
\centering
\subfloat[Filtar uz $b = 0.5$.]{
\label{fig:filter-bandwidth-05}
\includegraphics[width=4cm]{images/bandwidth05.jpg}
}
\hspace{50pt}
\subfloat[Filtar uz $b = 2$.]{
\label{fig:filter-bandwidth-2}
\includegraphics[width=4cm]{images/bandwidth2.jpg}
}
\caption{Gaborov filtar uz različite širine.}
\label{fig:filter-bandwidths}
\end{figure}

\chapter{Izlučivanje značajki pomoću Gaborovog filtra}

Izlučivanje značajki sa slika lica provodi se konvolucijom slike sa skupom
Gaborovih filtara. Ako je s $I(x, y)$ definirana slika sivih razina, odziv pojedinog Gaborov filtra i
slike računa se konvolucijom:
\begin{equation}
O(x,y,\lambda, \theta) = I(x,y) * g(x,y,\lambda, \theta),
\label{konvolucija-filter-slika}
\end{equation}
gdje je $g(x,y,\lambda, \theta)$ Gaborov filtar definiran parametrima $\lambda$
i $\theta$. Rezultantna slika koja nastaje je kompleksna slika s realnim i imaginarnim
dijelom. Parametri koji su u ovom radu pri tome korišteni su $\lambda =
\{2.5, 4, 5.6568, 8, 11.3137, 16\}$ i $\theta = \{ \frac{i \pi}{8} | i =
0 \ldots 7\}$ uz ostale parametre filtra $\gamma = 1$ i $b = \pi$ za koje se
pokazalo da su uspješni u izlučivanju značajki prilikom klasifikacije lica
\citep{shen2007gabor}.

Prije konvolucije Gaborov filter potrebno je normalizirati, odnosno Gaborovoj
funkciji oduzeti istosmjernu komponentu \engl{DC component}, kako odziv nebi bio
ovisan o varijacijama osvjetljenja izvorne slike, tj.~o istosmjernoj komponenti
izvorne slike. Nakon što se slika konvoluira s cijelim skupom Gaborovih filtera
dobiveni su odzivi $O(x,y,\lambda_i, \theta_j)$ za sve parametre $\theta$ i
$\lambda$. Za svaki element ($x,y$) slike kompleksnog odziva $O(x,y,\lambda_i,
\theta_j)$ računa se magnituda čime je dobivena nova slika realnih vrijednosti
magnitude pri čemu su vrijednosti magnitude unutar slike skalirane na interval
$[0, 1]$. Kako su izvorne slike, koje su korištene u raspoznavanju, dimenzija
$64\times64$ sliku magnituda potom svodimo na sliku manjih dimenzija, $8\times8$,
postupkom rezolucijske piramide: Slikovni element nove slike dobiven je
kao srednja vrijednost $8\times8$ susjednih slikovnih elemenata. Ako je s
$O^{r}(x,y,\lambda_i, \theta_j)$ označena dobivena slika magnituda nakon postupka
reduciranja dimenzije, svaka slika se tada pretvara u vektor $V(x,y,\lambda_i,
\theta_j)$ tako da se slijepe \engl{Concatenate} redovi slike
$O^{r}(x,y,\lambda_i, \theta_j)$. Dobiveni skup vektora se potom slijepi u
rezultantni vektor $X$:
\begin{equation}
 X = \left ( V(x,y,\lambda_1, \theta_1)^T \ldots
V(x,y,\lambda_i, \theta_j)^T \ldots V(x,y,\lambda_m, \theta_n)^T \right )
\end{equation}
gdje je s $^T$ označena operacija transponiranja.

Vektor $X$ je vektor značajki koje se koriste za raspoznavanja lica i on
je veličine $8\times8\times48 = 3072$  jer je nastao povezivanjem reduciranih
rezultata konvolucije izvorne slike s $48$ različitih Gaborovih filtara. Treba
primijetiti da je postupak reduciranja dimenzionalnosti nužan jer bi bez tog koraka
dobiveni vektor značajki bio dimenzija $64\times64\times48 = 196608$ što u
pogledu brzine izvođenja predstavlja ozbiljan problem prilikom računanja samog
vektora, a i prilikom koraka klasifikacije.




\chapter{Klasifikacija}

Prikaz izvorne slike vektorom realnih brojeva omogućava jednostavno korištenje
klasifikatora koji traže decizijske funkcije, tj.~granice u vektorskom prostoru.
Prilikom klasifikacije su tako isprobana dva pristupa: klasifikacija strojem s
potpornim vektorima \engl{Support vector machine} i klasifikacija metodom
$k$-najbližih susjeda \engl{$k$-nearest-neighbor algorithm}.


\section{Stroj s potpornim vektorim}
Izvorno, SVM traži optimalnu linearnu granicu, odnosno
hiperravninu, kako bi razdvojio različite razrede predstavljene skupom vektora u
vektorskome prostoru. Iskorištavanjem jezgrenog trika \engl{kernel trick} isti
klasifikator moguće je primijeniti za traženje proizvoljno nelinearne granice
između različitih razreda. Pri tome često korištena jezgrena funkcija je
radijalna bazna funkcija odnosno Gaussova jezgra:
\begin{equation}
k(\mathbf{x_i},\mathbf{x_j})=\exp(-\gamma \|\mathbf{x_i} - \mathbf{x_j}\|^2).
\end{equation}

Pokazano je da uz odabir ispravnih parametara \citep{keerthi2003asymptotic} linearni SVM
predstavlja specijalni slučaj SVM--a s radijalnom baznom funkcijom čime
se isključuje potreba za korištenjem linearnog SVM--a kao potencijalnog
klasifikatora. Prilikom korištenja radijalne bazne funkcije, a i općenito
SVM--a, prije samog postupka učenja podatke je potrebno skalirati kako bi utjecaj svih
atributa na klasifikaciju bio jednak, no u spomenutom slučaju klasifikacije
lica postupak skaliranja je izostavljen pod pretpostavkom da su područja slike s
većim vrijednostima odnosno većim rasponom vrijednosti od veće važnosti od
ostalih područja. 

Učenjem SVM--a traže se parametri pretpostavljenog oblika decizijske funkcije,
koja će ispravno klasificirati sve uzorke u skupu za učenje, što ponekad zbog
šuma u podacima ili njihove distribucije u prostoru nije moguće. Rješenje tog
problema nalazi se u korištenju SVM klasifikatora s mekim granicama definiranima
parametrom $C$ koji dozvoljava odstupanja od ispravne klasifikacije svih podataka
u skupu za učenje s ciljem bolje generalizacije nad još neviđenim skupom
podataka. Parametri SVM--a koji utječu na moć generalizacije nad neviđenim skupom
za ispitivanje su tako $C$ (parametar meke granice) i $\gamma$ (parametar
radijalne bazne funkcije). Oni zajedno čine prostor parametara čijom je pretragom
potrebno pronaći vrijednosti parametara, tj.~odabrati onaj model, koji će dati
SVM klasifikator s najmanjom pogreškom generalizacije.

Pretraga parametra odvija se odabirom modela s parametrima $(C, \gamma)$ iz skupa
$\left (C = {2^{-5}, 2^{-4}, \ldots , 2^{15}},  \gamma = {2^{-15}, 2^{-14},
\ldots, 2^3} \right )$ \citep{CC01a} koji daju najveću točnost klasifikacije u
procesu cross--validacije nad skupom za učenje. Točnost klasifikacije mjeri se formulom:
\begin{equation}
acc = \frac{n_c}{N},
\end{equation}
pri čemu je $n_c$ broj točno klasificiranih slika lica, a $N$ ukupan broj slika
lica nad kojima je provedeno testiranje. Nakon što su pronađeni parametri modela $(C,
\gamma)$, koji daju najveću točnost, klasifikator s navedenim parametrima ponovo
se uči na potpunom skupu za učenje. Parametri SVM klasifikatora korišteni u ovom
radu su $C = 2^{12}$ i $\gamma = 2^{-11}$.

\section{Metoda $k-nn$}

%TODO: 
% Popravi TEX gdje nije ok!
% Dodaj reference na literaturu, ja ovo pisem napamet i po slajdovima struce
% Obuci carape, vani je hladno
% Ra ra raspici, samo dobro opici...

Metoda $k-nn$ je jednostavna neparametarska metoda koja radi na principu
promatranja klasifikacije $k$ najbližih susjeda nepoznatog uzorka koji se
klasificira. Primjere za učenje moguće je predstaviti točkama u
$n$-dimenzionalnom prostoru $R^{n}$. Nepoznati uzorak se klasificira tako da
se u navedenom $n$-dimenzionalnom prostoru pronađe njemu $k$ najbližih susjeda
iz skupa za učenje te se uzorku dodijeli većinska klasifikacija od $k$ susjeda.
Za pronalazak najbližih susjeda koristi se euklidska metrika. Moguće je
korištenje i drugih metrika, ovisno o domeni problema.

Primjer $x$ opisan je vektorom značajki:

\begin{equation}
x = [x_1, x_2, ..., x_n]
\end{equation}

Euklidska udaljenost dva vektora $x$ i $y$ je:

\begin{equation}
d(x, y) = \sqrt{\sum_{i=1}^n |x_i - y_i|^2}
\end{equation}

Valja primjetiti da za $k-nn$ metodu nije potrebno izračunavati punu euklidsku
udaljenost. Kako je operacija kvadratnog korijena jedna od računalno skupljih
operacija zgodno je iskoristiti sljedeću relaciju:

\begin{equation}
a \leq b \Rightarrow a^2 \leq b^2 \quad \forall a,b \geq 0
\end{equation}

Vidimo da izračun kvadratnog korijena ne utječe na relativni odnos parova, što
je jedina bitna informacija za $k-nn$ metodu.

U algoritmu klasifikacije, primjeru $x_q$ pridjeljuje se klasifikacija koju ima najviše njegovih $k$ najbližih
susjeda, prema formuli:

\begin{equation}
f(x_q)=\arg\max_{v \epsilon V} \sum_{i=1}^k \delta(v,f(x_i))
\end{equation}

gdje je $\delta$ kroneckerov simbol (vrijednost je 1 ako su oba argumenta jednaka).


Lijene metode poput $k-nn$ klasifikatora konstruiraju različitu aproksimaciju
ciljne funkcije za svaki različiti novi uzorak koji treba biti klasificiran. Umjesto
procjene ciljne funkcije jednom za cijeli prostor, procjenjuje se ciljna funkcija
samo lokalno, u okolini novog uzorka što je pogodno u slučaju kompleksnih ciljnih
funkcija. Velika prednost u korištenju metode $k-nn$ je njezina jednostavnost
zbog toga što se sam postupak učenja svodi na pamćenje uzoraka i osim vrijednosti
$k$ nije potrebno podešavati nikakve dodatne parametre. Nedostatak je visoka
cijena klasificiranja svakog novog primjera jer se mora računati njegova
udaljenost od svih uzoraka u skupu za učenje.

Vrijednost parametra $k$ odabrana u ovom radu je $k = 3$ zbog činjenice da je u
skupu za učenje svaka osoba predstavljena s $5$ različitih slika lica pa
korištenje veće vrijednosti $k$ ne bi pridonjelo uspješnijoj klasifikaciji.

\chapter{Evaluacija}
\label{ch:eval}
Za evaluaciju korištena je baza slika lica veličine $64 \times 64$ piksela. Nad
slikama je provedena geometrijska i svjetlosna normalizacija. 

U bazi se nalazi ukupno 295 razreda (osoba) te 2360 uzoraka (slika) sa
jednolikom razdiobom (8 slika po razredu). Skup je podijeljen \emph{holdout} metodom
uzimajući 5 slika za učenje i 3 slike za testiranje.

U tablici \ref{tbl:eval} prikazana je uspješnost klasifikacije u ovisnosti o izboru
parametara Gaborovog filtera te klasifikatora\footnote{Za $k$--nn je dan samo
jedan očite manje uspješnosti i izrazito dužeg vremena izvršavanja.}. U tablici
se nalaze vrijednosti omjera dimenzija ($\gamma$) i širine filtra ($b$).
Orijentacija je fiksirana na $\theta = \{ \frac{i \pi}{8} | i =
0 \ldots 7\}$, a faze su $0$ i $\frac{\pi}{2}$. Za valnu duljinu su korištene
vrijednosti $\lambda = \{2.5, 4, 5.6568, 8, 11.3137, 16\}$
\citep{shen2007gabor}.

\begin{table}[ht]
\caption{Rezultati evaluacije}
\centering
\begin{tabular}{c c c c c c c}
\hline\hline
Klasifikator & $\gamma$ & $b$ & Ispravnih & Neispravnih &
Točnost & Trajanje (min) \\ [0.5ex]
\hline
$k$--nn & 0.5 & $\pi$ & 776 & 109 & 87.6\% & >120 \\
SVM & 0.5 & $\pi$ & 812  & 73 & 91.7\% & 23 \\
SVM & 0.7 & 1.5 & 823  & 62 & 92.9\% & 32\\
SVM & 0.5 & 1.5 & 823  & 62 & 92.9\% & 23\\
SVM & 1 & $\pi$ & 811  & 74 & 91.6\% & 20\\
SVM & 0.7 & $\pi$ & 812  & 73 & 91.7\% & 20\\
SVM & 1 & 1.5 & 822  & 63 & 92.8\% & 36\\ [1ex]
\hline
\end{tabular}
\label{tbl:eval}
\end{table}

\chapter{Zaključak}

Korištenje Gaborovog filtra ima dobru podlogu u biološkim osnovama vida
sisavaca te se očekivalo da će se njegovom upotrebom u klasifikaciji lica
postići zadovoljavajući rezultati. Uspješnost raspoznavanja lica od $p\%$
dobivena u ovom radu potkrjepljuje navedena očekivanja. Iako su rezultati
odlični, korišteni parametri i oblik izlučenih značajki predstavljaju
samo jedan od mnogobrojnih mogućih načina korištenja Gaborovog filtra te je
moguće da se korištenjem drugih parametara i drugačijim predstavljanjem
izlučenih značajki mogu postići još bolji rezultati. Upravo u navedenoj
činjenici vidi se prilika za daljnji nastavak istraživanja primjene Gaborovog
filtra u izradi sustava za računalno raspoznavanje lica.

\bibliography{literatura}
\bibliographystyle{plainnat}

\end{document}
