\documentclass{article}
\usepackage[utf8]{inputenc} \usepackage[croatian]{babel} \usepackage[T1]{fontenc}
\usepackage{lmodern}
\usepackage{algorithmic}
\usepackage{algorithm}
\usepackage{longtable}
\usepackage{graphicx}
\usepackage{booktabs}
\usepackage{hyperref}
% Da bi se promjenio stil citiranja umjesto: [authoryear, round] staviti:
% [numbers, square]
\usepackage[authoryear, round]{natbib}
\usepackage{amsmath}
\usepackage{subfig}
\usepackage{fixltx2e}
\usepackage{todo}
\usepackage{url}
\usepackage{textcomp}
\usepackage{float}

\newcommand{\engl}[1]{(engl.~\emph{#1})}

\begin{document}
\title{Izlučivanje značajki lica Gaborovim filterom}
\author{Tomislav Reicher \and Krešimir Antolić \and Igor Belša \and Marko Ivanković \and Ivan Krišto \and Maja Legac \and Tomislav Novak}
\date{20.01.2010.}
\maketitle

\tableofcontents

\section{Uvod}
Raspoznavanje uzoraka je znanstvena disciplina iz područja računarskih znanosti
čiji je cilj klasifikacija ili razvrstavanje objekata u jedan od brojnih razreda
ili klasa. Iako su područja uporabe brojna u ovom radu koncentracija je na
raspoznavanju vizualnih uzoraka. Točnije, radi se o raspoznavanju lica.

Raspoznavanje lica uključuje računalno prepoznavanje indentiteta na temelju
značajki dobivenih obradom slike lica. Iako ljudima lak zadatak, prepoznavanje
lica i njihovo raspoznavanje je veoma zahtjevan posao za računalo. Zadatak
postaje tim zahtjevniji ako su lica slikana pod različitim osvijetljenjem,
različitim kutovima ili osobe na slikama imaju različite izraze lica.

Kao osnovna motivacija za korištenje Gaborovog filtera za izvlačenje značajki je
veza sa biološkim osobinama vida kod sisavaca čiji su receptori osjetljivi na
karakteristične prostorne frekvencije te njihovu orijentaciju. Gaborov filter
može iskoristit vizualne osobine kao što su lokalizacija prostora i selekcija
orijentacije pojedine prostorne frekvencije
\citep{bhuiyan2007onfacerecognition}\nocite{daugman1985uncertainty}.

% Ovo je opis o čemu koji odjeljak govori. Ovo se MORA nalaziti na kraju uvoda.
U 2.~odjeljku ukratko je opisana osnovna ideja korištenja Gaborovog
filtra u izlučivanju značajki za klasifikaciju lica, u 3.~odjeljku prikazan je
matematički model dvodimenzionalnog Gaborovog filtra te su navodena objašnjenja
i utjecaj pojedinih parametara Gaborovog filtra. U 4.~odjeljku opisan je
postupak izlučivanja značajki Gaborovim filtrom te su navedeni korišteni
parametri. U 5. odjeljku opisana je metoda kojom su lica klasificirana na temelju
izlučenih značajki, dok je provedena evaluacija opisana u poglavlju ?. Konačno, na
kraju je dan zaključak.

\section{Osnovna ideja}
Slike lica osoba koje računalni sustav treba raspoznati podložne su različitom
šumu kao što su varijacija osvjetljenja, položaj lica, izražaji lica (lice s
osmijehom, ozbiljno lice), brkovi i brada kod lica muškaraca, naočale itd. Jedna
od metoda izlučivanja značajki koja se pritom koristi kako bi se doskočilo
navedenim problemima je izlučivanje značajki Gaborovim filtrom. Osnovna ideja u
izlučivanju značajki Gaborovim filterom je pronaći značajke lica karakterizirane
prostornom frekvencijom \engl{Spatial frequency}, prostornom lokalizacijom
\engl{Spatial locality} i orijentacijom koje će predstavljati diskriminatorne
značajke za raspoznavanje lica različitih osoba. Promatranje slike u
frekvencijskoj domeni \engl{Frequency domain}, gdje je cijela slika predstavljena
skupom različitih prostornih frekvencija, omogućuje lakšu manipulaciju s
različitim informacijama sa slike. Uklanjanjem nekog manjeg skupa frekvencija i
prikazom dobivene slike moguće je ukloniti različiti šum na slici. Problem kod
prikaza u frekvencijskoj domeni je globalni karakter prikaza, tj. za
pojedinu frekvenciju nije moguće odrediti gdje se prostorno na slici ona pojavila.
Moguće je samo utvrditi da ona na slici postoji. Gaborov filter rješava
upravo navedeni problem prostorne lokalizacije. Pomoću njega možemo efikasno
odrediti je li se u pojedinom dijelu slike pojavila određena frekvencija, koja
je amplituda i faza te frekvencije te koji je smjer njenog rasprostiranja.

Nakon izlučivanja skupa značajki pomoću Gaborovog filtra lice sa slike raspoznaje
se pomoću stroja s potpornim vektorima \engl{Support vector machine,
SVM} s radijalnom baznom funkcijom.

\section{Gaborov filter}

Jezgra Gaborovog filtra dvodimenzionalna je Gaborova funkcija koju čini
kompleksni sinusoidalni val moduliran Gaussovom funkcijom. Gaborova funkcija
dana je kao \citep{petkovgabor}:
\begin{equation}
% g(x,y)=\frac{1}{2\pi \sigma^2_{xy}}e^{-\left ( \frac{x'^2 +
% y'^2}{2\sigma^2_{x,y}} \right)} \times \left ( e^{2\pi i r_0 x'} -
% e^{-\frac{r_0^2}{2\sigma^2_{uv}}}\right),
g_{\lambda,\theta,\sigma,\gamma}(x,y) = \exp\left ( -
\frac{x'^2+\gamma^2 y'^2}{2\sigma^2}\right ) \exp \left ( j2\pi
\frac{x'}{\lambda} \right ),
\label{2d-gabor}
\end{equation}
pri čemu su
\begin{eqnarray*}
x' = x \cos \theta + y \sin \theta, \\
y' = -x \sin \theta + y \cos \theta.
\end{eqnarray*}

Ovdje $\sigma$ označava standardnu devijaciju Gaussove funkcije u $x$ smjeru,
$\gamma$ označava omjer devijacije Gaussove funkcije u $x$ i $y$ smjeru ($\gamma =
\frac{\sigma_x}{\sigma_y}$), $\lambda$ je valna duljina potpornog sinusoidalnog
vala, a $\theta$ kut smjera rasprostiranja sinusoidalnog vala.

Frekvencija i odabir orijentacije Gaborovog filtra su izražajnije u
domeni frekvencijskog prikaza predstavljenog Gaborovom funkcijom (\ref{gabor-frek}) koja
određuje koliko filter utječe na svaku frekvencijsku komponentu ulazne slike.
\begin{equation}
G(u,v) = \exp \left ( - \frac{(u-u_0)^2 + (v-v_0)^2}{2\sigma^2_{uv}}\right ),
\label{gabor-frek}
\end{equation}
\begin{equation}
\sigma_{uv} = \frac{1}{2\pi \sigma}.
\end{equation}
Parametri ($u_0$, $v_0$) definiraju prostornu frekvenciju sinusoidalnog vala u
ravnini koja također može biti izražena polarnim koordinatama kao radijalna
frekvencija $r_0$, odnosno valna duljina $\lambda$, i orijentacija $\theta$ čime
dobivamo parametre ovisne o frekvenciji ekvivalentne onima iz Gaborove funkcije u prostornoj
domeni:
\begin{eqnarray}
r_0 = \sqrt{u_0^2 + v_0^2}, \\
\tan \theta = \frac{v_0}{u_0}, \\
\lambda = \frac{1}{r_0}.
\end{eqnarray}

Iz gornje formule možemo vidjeti da je Gaborova funkcija u frekvencijskoj domeni
jednaka Gaussovoj funkciji postavljenoj u točci $(u_0, v_0)$, odnosno $(r_0,
\theta)$, te s time na umu Gaborov filter možemo gledati kao pojasno propusni
filter koji propušta frekvencije u malom pojasu oko centralne frekvencije $(u_0,
v_0)$.

Kako bi bilo moguće odabrati parametre Gaborovog filtra koji će služiti
za izlučivanje značajki potrebno je promotriti utjecaj pojedinih parametara na
Gaborovu funkciju.

\subsection{Valna duljina ($\lambda$)}
Valna duljina se odnosi na valnu duljinu sinusoidalnog vala u Gaborovoj funkciji
kojom se ujedino definira i prostorna frekvencija na koju će sam Gaborov filter
biti osjetljiv. Valna duljina ovdje određuje duljinu ciklusa u slikovnim
elemetima \engl{Pixel} i realan je broj veći ili jednak $2$. Prema
Nyquist--Shannonovom teoremu uzorkovanja signal koji sadrži frekvencije veće od
polovine frekvencije uzorkovanja ne može biti rekonstruiran u potpunosti, stoga
je gornja granica frekvencije koju $2D$ slika može sadržavati jednaka $F_{max} =
0.5$ ciklusa/slikovnom elementu, a to je upravo znači da će valna duljina
biti veća ili jednaka od $2$ slikovna elementa.

Slike realnih dijelova Gaborovih filtera uz parametre $\theta = 0,\: b
= 1$, i $\gamma = 0.5$ mogu se vidjeti na slici \ref{fig:filter-wavelengths}.

\begin{figure}[h!tb]
\centering
\subfloat[Filter uz $\lambda = 5$.]{
\label{fig:filter-lambda-5}
\includegraphics[width=4cm]{images/wavelength5.jpg}
}
\hspace{50pt}
\subfloat[Filter uz $\lambda = 10$.]{
\label{fig:filter-lambda-10}
\includegraphics[width=4cm]{images/wavelength10.jpg}
}
\caption{Gaborov filter uz različite valne duljine.}
\label{fig:filter-wavelengths}
\end{figure}

% TODO: Pogledati o čemu se ovdje radi\ldots Gledao sam formulu, ništa pametno
% nisam zaključio, generirao sam slike, također ništa pametno\ldots Nešto mi
% promiče\ldots Zakomentirao sam tu rečenicu dok mi ne postane smislena.
%Uz $\lambda = 2$, ne bi se smjela koristiti kombinacija $\varphi = \pm 90$ jer
%u tom slučaju Gaborova funkcija je uzrokovana u svojim
%nul--prijelazima\TODO{``sampled in its zero crossings''?}.


\subsection{Orijentacija ($\theta$)}
Orijentacija određuje smjer rasprostiranja moduliranog sinusnog signala odnosno
kut normale paralelnih pruga Gaborovog filtera. Za potpuno
specificiranje prostorne frekvencije nije dovoljna informacija o valnoj
duljini, već je potrebno odrediti i smjer u kojem se ta frekvencija
rasprostire. Određena je kutom od $0$ do $360$ stupnjeva odnosno $0$ do $2\pi$
radijana. Prikaze realnog dijela Gaborovog filtera sa orijentacijama od
45\textdegree, 80\textdegree\,i 0\textdegree\, možete vidjeti na slikama
\ref{fig:filter-orientation-45}, \ref{fig:filter-orientation-80} i \ref{fig:filter-lambda-10}.


\begin{figure}[h!tb]
\centering
\subfloat[Filter uz $\theta = 45$\textdegree.]{
\label{fig:filter-orientation-45}
\includegraphics[width=4cm]{images/orientation45.jpg}
}
\hspace{50pt}
\subfloat[Filter uz $\theta = 80$\textdegree.]{
\label{fig:filter-orientation-80}
\includegraphics[width=4cm]{images/orientation80.jpg}
}
\caption{Gaborov filter uz različite orijentacije.}
\label{fig:filter-orientations}
\end{figure}


\subsection{Omjer dimenzija ($\gamma $)}
Parametar koji se preciznije naziva prostorni omjer dimenzija, određuje
eliptičnost Gaborove funkcije tj. odnos između devijacija Gaussove funkcije u
$x$ i $y$ smjeru. Za $\gamma = 1$ eliptičnost se svodi na krug, dok je za
$\gamma < 1$ funkcija je izdužena u smjeru paralelnom s paralelnim prugama
funkcije. Primjer filtera sa različitim omjerima dimenzija se može vidjeti na
slici \ref{fig:filter-ratios}.

\begin{figure}[h!tb]
\centering
\subfloat[Filter uz $\gamma = 0.5$.]{
\label{fig:filter-gamma-05}
\includegraphics[width=4cm]{images/wavelength10.jpg}
}
\hspace{50pt}
\subfloat[Filter uz $\gamma = 1$.]{
\label{fig:filter-gamma-1}
\includegraphics[width=4cm]{images/ratio1.jpg}
}
\caption{Gaborov filter uz različite omjere dimenzija.}
\label{fig:filter-ratios}
\end{figure}

\subsection{Standardna devijacija ($\sigma$)}
Standardna devijacija Gaussove funkcije utječe na dio prostora u kojem će se
računati odziv Gaborovog filtra, a jednako tako utječe i na širinu
frekvencijskog pojasa koji će gaborov filter propustiti, stoga je poželjno da
omjer standardne devijacije i radijalne frekvencije Gaborove funkcije bude
konstantan. Gaborova funkcija s većom frekvencijom će tako imati
manju standarnu devijaciju te time pokrivati manju površinu u prostornoj domeni
od gaborove funkcije s manjom frekvencijom. 

Širina filtra $b$ Gaborovog filtra povezana je s omjerom
$\frac{\sigma}{\lambda}$, ($\lambda = \frac{1}{r_0}$) i definirana s:
\begin{eqnarray}
b = \log_2{\left ( \frac{\frac{\sigma}{\lambda}\pi + \sqrt{\frac{\ln2}{2}}}
{\frac{\sigma}{\lambda}\pi - \sqrt{\frac{\ln2}{2}}} \right )}, \\
\frac{\sigma}{\lambda} =
\frac{1}{\pi}\sqrt{\frac{ln2}{2}}\cdot\frac{2^b+1}{2^b-1}.
\end{eqnarray}
Prilikom izvedbe filtra vrijednost $\sigma$ se ne može direktno
odrediti, nego se mijenja samo preko vrijednosti širine filtera,
$b$. Uz zadani $b$ i poznati $\lambda$, $\sigma$ se određuje prema gornjoj
formuli.
 
Širina filtera zadana je kao pozitivni realni broj. Što je širina manja,
$\sigma$ je veća i povećava se broj naglašavajućih i prigušavajućih pruga
Gaborovog filtera.~ Primjere realnog dijela filtra s
različitim širinama moguće je vidjeti na slikama \ref{fig:filter-bandwidth-05}
i \ref{fig:filter-bandwidth-2}. Na prethodnim primjerima korištena je širina
filtera od $b = 1$.

\begin{figure}[h!tb]
\centering
\subfloat[Filter uz $b = 0.5$.]{
\label{fig:filter-bandwidth-05}
\includegraphics[width=4cm]{images/bandwidth05.jpg}
}
\hspace{50pt}
\subfloat[Filter uz $b = 2$.]{
\label{fig:filter-bandwidth-2}
\includegraphics[width=4cm]{images/bandwidth2.jpg}
}
\caption{Gaborov filter uz različite širine.}
\label{fig:filter-bandwidths}
\end{figure}

\section{Izlučivanje značajki pomoću Gaborovog filtra}

Kako bi izlučili značajke sliku konvoluiramo sa skupom Gaborovih filtera. Ako
je s $I(x, y)$ definirana slika sivih razina, odziv pojedinog Gaborovog filtra i
slike računa se konvolucijom:
\begin{equation}
O(x,y,\lambda, \theta) = I(x,y) * g(x,y,\lambda, \theta),
\label{konvolucija-filter-slika}
\end{equation}
gdje je $g(x,y,\lambda, \theta)$ Gaborov filter definiran parametrima $\lambda$
i $\theta$. Parametri koji su u ovom radu pri tome korišteni su $\lambda =
\{2.5, 4, 5.6568, 8, 11.3137, 16\}$ i $\theta = \{ \frac{i \pi}{8} | i =
0 \ldots 7\}$ uz ostale parametre filtra $\gamma = 1$ i $b = \pi$ za koje se
pokazalo da su uspješni u izlučivanju značajki prilikom klasifikacije lica
\citep{shen2007gabor}.

Rezultantna slika koja nastaje je kompleksna slika s realnim i imaginarnim
dijelom. Prije konvolucije Gaborov filter potrebno je normalizirati, odnosno
oduzeti mu istosmjernu komponentu \engl{DC component}, kako odziv nebi bio ovisan
o varijacijama osvjetljenja izvorne slike, tj. o istosmjernoj komponenti izvorne
slike. Nakon što se slika konvoluira s cijelim skupom Gaborovih filtera dobiveni
su odzivi $O(x,y,\lambda_i, \theta_j)$ za sve parametre $\theta$ i $\lambda$. Za
svaki element ($x,y$) slike kompleksnog odziva $O(x,y,\lambda_i, \theta_j)$
računa se magnituda čime je dobivena nova slika realnih magnituda. Kako su
izvorne slike korištene u raspoznavanju dimenzija $64\times64$ sliku magnituda potom
svodimo na sliku manjih dimenzija, $8\times8$, postupkom rezolucijske piramide
tako da je slikovni element nove slike dobiven kao srednja vrijednost $8\times8$
susjednih slikovnih elemenata. Ako je s $O^{r}(x,y,\lambda_i, \theta_j)$
označena dobivena slika magnituda nakon postupka reduciranja dimenzije, svaka
slika se tada pretvara u vektor $V(x,y,\lambda_i, \theta_j)$ tako da se
slijepe \engl{Concatenate} redovi slike $O^{r}(x,y,\lambda_i, \theta_j)$.
Dobiveni skup vektora se potom slijepi u rezultantni vektor $X$:
\begin{equation}
 X = \left ( V(x,y,\lambda_1, \theta_1)^T \ldots
V(x,y,\lambda_i, \theta_j)^T \ldots V(x,y,\lambda_m, \theta_n)^T \right )
\end{equation}
Vektor $X$ je vektor značajki koje se koriste za raspoznavanja lica i on
je veličine $8\times8\times48 = 3072$  jer je nastao povezivanjem reduciranih
rezultata konvolucije izvorne slike s $48$ različitih Gaborovih filtara. Treba
primjetiti da je postupak reduciranja dimenzionalnosti nužan jer bi bez tog koraka
dobiveni vektor značajki bio dimenzija $64\times64\times48 = 196608$ što u
pogledu brzine izvođenja predstavlja ozbiljan problem prilikom računanja samog
vektora, a i prilikom koraka klasifikacije.


\section{Klasifikacija}

Prikaz izvorne slike vektorom realnih brojeva omogućava jednostavno korištenje
klasifikatora koji traže decizijske funkcije, tj.~granice u vektorskom
prostoru.
U radu je za klasifikaciju korišten stroj s potpornim vektorima \engl{Support
vector machine}. Izvorno, SVM traži optimalnu linearnu granicu, odnosno
hiperravninu, kako bi razdvojio različite razrede predstavljene skupom vektora u
vektorskome prostoru. Iskorištavanjem jezgrenog trika \engl{kernel trick} isti
klasifikator moguće je primijeniti za traženje proizvoljno nelinearne granice
između različitih razreda. Pri tome često korištena jezgrena funkcija je
radijalna bazna funkcija odnosno Gaussova jezgra:
\begin{equation}
k(\mathbf{x_i},\mathbf{x_j})=\exp(-\gamma \|\mathbf{x_i} - \mathbf{x_j}\|^2).
\end{equation}

Pokazano je da uz odabir ispravnih parametara \citep{keerthi2003asymptotic} linearni SVM
predstavlja specijalni slučaj SVM--a s radijalnom baznom funkcijom čime
se isključuje potreba za korištenjem linearnog SVM--a kao potencijalnog
klasifikatora. Prilikom korištenja radijalne bazne funkcije, a i općenito
SVM--a, prije samog postupka učenja podatke je potrebno skalirati kako bi utjecaj svih
atributa na klasifikaciju bio jednak, no u spomenutom slučaju klasifikacije
lica postupak skaliranja je izostavljen pod pretpostavkom da su područja slike s
većim vrijednostima odnosno većim rasponom vrijednosti od veće važnosti od
ostalih područja. 

Učenjem SVM--a traže se parametri pretpostavljenog oblika decizijske funkcije,
koja će ispravno klasificirati sve uzorke u skupu za učenje, što ponekad zbog
šuma u podacima ili njihove distribucije u prostoru nije moguće. Rješenje tog
problema nalazi se u korištenju SVM klasifikatora s mekim granicama definiranima
parametrom $C$ koji dozvoljava odstupanja od ispravne klasifikacije svih podataka
u skupu za učenje s ciljem bolje generalizacije nad još neviđenim skupom
podataka. Parametri SVM--a koji utječu na moć generalizacije nad neviđenim skupom
za ispitivanje su tako $C$ (parametar meke granice) i $\gamma$ (parametar
radijalne bazne funkcije). Oni zajedno čine prostor parametara čijom je pretragom
potrebno pronaći vrijednosti parametara, tj.~odabrati onaj model, koji će dati
SVM klasifikator s najmanjom pogreškom generalizacije.

Pretraga parametra odvija se odabirom modela s parametrima $(C, \gamma)$ iz skupa
$\left (C = {2^{-5}, 2^{-4}, \ldots , 2^{15}},  \gamma = {2^{-15}, 2^{-14},
\ldots, 2^3} \right )$ \citep{CC01a} koji daju najveću točnost klasifikacije u
procesu cross--validacije nad skupom za učenje. Točnost klasifikacije mjeri se formulom:
\begin{equation}
acc = \frac{n_c}{N},
\end{equation}
pri čemu je $n_c$ broj točno klasificiranih slika lica, a $N$ ukupan broj slika
lica nad kojima je provedeno testiranje. Nakon što su pronađeni parametri modela $(C,
\gamma)$, koji daju najveću točnost, klasifikator s navedenim parametrima ponovo
se uči na potpunom skupu za učenje. Parametri SVM klasifikatora korišteni u ovom
radu su $C = 2^{12}$ i $\gamma = 2^{-11}$.


\section{Zaključak}

Korištenje Gaborovog filtra ima dobru podlogu u biološkim osnovama vida
sisavaca te se očekuje da će se njegovom upotrebom, korištenjem različitih klasifikatora i
automatiziranom optimizacijom parametara postići uspjeh u raspoznavanju lica različitih osoba.

\bibliography{literatura}
\bibliographystyle{plainnat}

\end{document}
